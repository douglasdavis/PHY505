\documentclass[aps,prd,twocolumn,nofootinbib]{revtex4-1}
\usepackage{amsmath}
\usepackage{graphicx}
\usepackage{subfig}
\usepackage{epsfig}
\usepackage{listings}
\usepackage[hidelinks,hyperfootnotes=false,bookmarks=false]{hyperref}
\usepackage[colorinlistoftodos]{todonotes}
\usepackage{verbatim}
\usepackage{float}
\begin{document}
\title{A Survey of Sterile Neutrinos}
\author{Douglas Davis}
\author{Matthew Epland}
\author{Justin Raybern}
\author{Pingchuan Zhao}
\affiliation{Department of Physics, Duke University, Durham, NC 27707, USA}
\date{\today}
\begin{abstract}
xxxxxxxxxxxxxxxxxxxxxxxxxxxxxxxxx xxxxxxxxxxxxxxxxxxxxxxxxxxxxxxxxxxxxxxxxxx xxxxxxxxxxxxxxxxxxxxxxxxxxxxxxxxxxxxxxxxxxxxxxxxxxxxxxxxxxxxxxxxxxxxxxxxxxx xxxxxxxxxxxxxxxxxxxxxxxxxxxxxxxxxxxxxxxxxx xxxxxxxxxxxxxxxxxxxxxxxxxxxxxxxxxxxx xxxxxxxxxxxxxxxxxxxxxxxxxxxxxxxxxxxxxxxxxx xxxxxxxxxxxxxxxxxxxxxxxxxxxxxxxxxxxxxxxxxx xxxxxxxxxxxxxxxxxxxxxxxxxxxxxxxxxxxxx xxxxxxxxxxxxxxxxxxxxxxxxxxxxxxxxxxxxxxxxxx xxxxxxxxxxxxxxxxxxxxxxxxxxxxxxxxxxxxxxxxxx xxxxxxxxxxxxxxxxxxxxx
\end{abstract}\maketitle
\section{Theory}
\label{sec:theory}
xxxxxxxx xxxxxx xxxxxxxx xxxxxx xxxxxxxx xxxxxx xxxxxxxx xxxxxx xxxxxxxx xxxxxx xxxxxxxx xxxxxx xxxxxxxx xxxxxx
\subsection{Theory Ping}
xxxxxxxx xxxxxx xxxxxxxx xxxxxx xxxxxxxx xxxxxx xxxxxxxx xxxxxx xxxxxxxx xxxxxx xxxxxxxx xxxxxx xxxxxxxx xxxxxx
\subsection{Theory Justin}
xxxxxxxx xxxxxx xxxxxxxx xxxxxx xxxxxxxx xxxxxx xxxxxxxx xxxxxx xxxxxxxx xxxxxx xxxxxxxx xxxxxx xxxxxxxx xxxxxx
\section{Previous Experimental Efforts}
There exists considerable experimental evidence for the existence of a fourth neutrino at energy scales in the eV range. The sources for evidence include the short baseline neutrino oscillation experiments LSND (Liquid Scintillator Neutrino Detector) at Los Alamos~\cite{LSND}, and MiniBooNE (Mini Booster Neutrino Experiment) at Fermilab~\cite{mini1,mini2}. There is also the reactor neutrino anomaly~\cite{reactor_anom1}, and also in the radioactive source calibrations for the solar neutrino experiments GALLEX~\cite{gallex1,gallex2} and SAGE~\cite{sage1,sage2}.
\subsection{LSND}
The LSND experiment was developed to search for $\overline{\nu}_\mu \rightarrow \overline{\nu}_e$ oscillations. The source of $\overline{\nu}_\mu$ was the decay at rest (DAR) of $\mu^+$. LSND used a proton beam on a target to produce pions. The decay chain was as follows:
\begin{align}
  \begin{split}
    \pi^+ &\rightarrow \mu^+  \nu_\mu, \\
    \mu^+ &\rightarrow e^+  \nu_e  \overline{\nu}_\mu.
  \end{split}
\end{align}
The $\overline{\nu}_e$ events were detected through the interaction:
\begin{align}
  \overline{\nu}_e  p \rightarrow e^+ n,
\end{align}
inside of a large Cherenkov detector filled with liquid scintillator, utilizing 1280 phototubes. Figure~\ref{fig:lsnd_miniboone}(L) shows the excess of $\overline{\nu}_e$ events observed by LSND. The observation was $87.9\pm 22.4\pm6$ events over expected background. Their final result claimed the existence of $\overline{\nu}_\mu\rightarrow\overline{\nu}_e$ oscillations at $\Delta m^2 \sim 1\text{ eV}$ with 3.8$\sigma$. This is in heavy conflict with the existing three light flavor neutrino model supported by atmospheric and solar neutrino oscillation experiments, as well as measurements at LEP predicting $N=2.984\pm0.0082$ for weakly interacting neutrino flavors. Therefore, at least one ``sterile'' neutrino is required.
\begin{figure*}
  \includegraphics[width=1\textwidth]{../figures/lsnd_miniboone.pdf}
  \caption{(L) ... (C) ... (R) ...}
  \label{fig:lsnd_miniboone}
\end{figure*}
\subsection{MiniBooNE Doug}
The MiniBooNE experiment was developed to answer the LSND anomaly.
\subsection{Reactor Anomaly}
xxxxxxxx xxxxxx xxxxxxxx xxxxxx xxxxxxxx xxxxxx xxxxxxxx xxxxxx xxxxxxxx xxxxxx xxxxxxxx xxxxxx xxxxxxxx xxxxxx
\section{Current and Future Experimental Efforts}
yep
xxxxxxxx xxxxxx xxxxxxxx xxxxxx xxxxxxxx xxxxxx xxxxxxxx xxxxxx xxxxxxxx xxxxxx xxxxxxxx xxxxxx xxxxxxxx xxxxxx
\subsection{Current Matt}
xxxxxxxx xxxxxx xxxxxxxx xxxxxx xxxxxxxx xxxxxx xxxxxxxx xxxxxx xxxxxxxx xxxxxx xxxxxxxx xxxxxx xxxxxxxx xxxxxx
experiment
\subsection{Future Matt}
xxxxxxxx xxxxxx xxxxxxxx xxxxxx xxxxxxxx xxxxxx xxxxxxxx xxxxxx xxxxxxxx xxxxxx xxxxxxxx xxxxxx xxxxxxxx xxxxxx
is cool

\section{Conclusions}
xxxxxxxx xxxxxx xxxxxxxx xxxxxx xxxxxxxx xxxxxx xxxxxxxx xxxxxx xxxxxxxx xxxxxx xxxxxxxx xxxxxx xxxxxxxx xxxxxx
who cares

\begin{thebibliography}{9}
\bibitem{LSND}
  A.~Aguilar-Arevalo \emph{et al.} Evidence for neutrino oscillations from the observation of anti-neutrino(electron) appearance in a anti-neutrino(muon) beam. \emph{Phys. Rev. D.}, {\bf 64} 112007, 2001.
\bibitem{mini1}
  A.~Aguilar-Arevalo \emph{et al.} Event Excess in the MiniBooNE Search for $\nu_\mu \rightarrow \nu_e$ Oscillations. \emph{Phys. Rev. Lett.}, {\bf 105} 181801, 2010.
\bibitem{mini2}
  A.~Aguilar-Arevalo \emph{et al.} Improved Search $\nu_\mu \rightarrow \nu_e$ Oscillations in the MiniBooNE Experiment. \emph{Phys. Rev. Lett.}, {\bf 110} 161801, 2013.
\bibitem{reactor_anom1}
  G. Mention, M. Fechner, Th. Lasserre, Th.A. Mueller, D. Lhuillier, \emph{et al.} The Reactor Antineutrino Anomaly. \emph{Phys. Rev. D.}, {\bf 83} 073006, 2011.
\bibitem{reactor_anom2}
  Th. A. Mueller, D. Lhuillier, M. Fallot, A. Letourneau, S. Cormon, \emph{et al.} Improved Predictions of Reactor Antineutrino Spectra. \emph{Phys. Rev. C.}, {\bf 83} 054615, 2011.
\bibitem{gallex1}
  P.~Anselmann~\emph{et al.} First results from the Cr-51 neutrino source experiment with the GALLEX detector. \emph{Phys. Lett. B.}, {\bf 342} 440-450, 1995
\bibitem{gallex2}
  W.~Hampel~\emph{et al.} Final results of the Cr-51 neutrino source experiments in GALLEX. \emph{Phys. Lett. B.}, {\bf 420} 114-126, 1998.
\bibitem{sage1}
  Dzh.~N.~Abdurashitov, V.N.~Gavrin, S.V. Girin, V.V. Gorbachev, Tatiana V. Ibragimova, \emph{et al.} The Russian-American gallium experiment (SAGE) Cr neutrino source measurement. \emph{Phys. Rev. Lett.}, {\bf 77} 4708-4711, 1996.
\bibitem{sage2}
  J.~N.~Abdurashitov~\emph{et al.} Measurement of the response of the Russian-American gallium experiment to neutrinos from a Cr-51 source. \emph{Phys. Rev. C.}, {\bf 59} 2246-2263, 1999.

\end{thebibliography}

\end{document} %%% end of doc %%%
