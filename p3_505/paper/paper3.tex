\documentclass[aps,prd,twocolumn,nofootinbib]{revtex4-1}
\usepackage{amsmath}
\usepackage{graphicx}
\usepackage{subfig}
\usepackage{epsfig}
\usepackage{listings}
\usepackage[hidelinks,hyperfootnotes=false,bookmarks=false]{hyperref}
\usepackage[colorinlistoftodos]{todonotes}
\usepackage{verbatim}
\usepackage{float}
\begin{document}
\title{A Survey of Sterile Neutrinos}
\author{Douglas Davis}
\author{Matthew Epland}
\author{Justin Raybern}
\author{Pingchuan Zhao}
\affiliation{Department of Physics, Duke University, Durham, NC 27707, USA}
\date{\today}
\begin{abstract}
  xxxxxxxxxxxxxxxxxxxxxxxxxxxxxxxxx xxxxxxxxxxxxxxxxxxxxxxxxxxxxxxxxxxxxxxxxxx xxxxxxxxxxxxxxxxxxxxxxxxxxxxxxxxxxxxxxxxxxxxxxxxxxxxxxxxxxxxxxxxxxxxxxxxxxx xxxxxxxxxxxxxxxxxxxxxxxxxxxxxxxxxxxxxxxxxx xxxxxxxxxxxxxxxxxxxxxxxxxxxxxxxxxxxx xxxxxxxxxxxxxxxxxxxxxxxxxxxxxxxxxxxxxxxxxx xxxxxxxxxxxxxxxxxxxxxxxxxxxxxxxxxxxxxxxxxx xxxxxxxxxxxxxxxxxxxxxxxxxxxxxxxxxxxxx xxxxxxxxxxxxxxxxxxxxxxxxxxxxxxxxxxxxxxxxxx xxxxxxxxxxxxxxxxxxxxxxxxxxxxxxxxxxxxxxxxxx xxxxxxxxxxxxxxxxxxxxx
\end{abstract}\maketitle
\section{Theory}
\label{sec:theory}
xxxxxxxx xxxxxx xxxxxxxx xxxxxx xxxxxxxx xxxxxx xxxxxxxx xxxxxx xxxxxxxx xxxxxx xxxxxxxx xxxxxx xxxxxxxx xxxxxx
\subsection{Theory}
%ping
A neutrino is an electrically neutral, weakly interacting elementary particle with half integer spin. There are generations of neutrinos: electron-neutrino, muon-neutrino and tau-neutrino. It is first indicated from beta decay spectrum. From the experiment of radioactive decay, the outgoing electron energy varies differently, hence it is not a 2-body decay, but actually a 3-body decay. Pauli first called that small, neutral charge particle neutrino. Soon, this explanation is accepted among physicists and after a few years, anti-neutrinos is found from inverse beta decay. 

In the Standard Model, there are three generations of neutrinos. However, the Lagrangian has no neutrino mass term, giving neutrinos have exactly zero mass. Also, neutrinos and anti-neutrinos are distinct, separating by helicity. From experiment measurements, neutrinos always have left helicity and anti-neutrinos with right helicity. Neutrinos couple with W,Z bosons only in this model~\cite{ping1}.

Although the Standard Model is exceedingly successful in explaining the interactions among fundamental particles, it can not be the ultimate theory, since it does not give explanation on gravity, dark matter, neutrino mass and neutrino oscillation. In 1960s, the solar neutrino experiment first indicates that there is a change of flavor when neutrinos traveling. This is also the first evidence that shows neutrinos have some mass. By using the detector Kamiokande and later the upgraded version, Super Kamiokande, they observe the flavor changes not only from solar neutrinos but also long-baseline reactor neutrinos. 

Starting in the late 1960s, there is a discrepancy using Standard Model to give explanations to neutrinos arriving from the sun~\cite{ping2}. The observed electron neutrinos are only between one third to one half of what Standard Model predicts. This problem remains unsolved until almost 30 years later as the observation of neutrino oscillation, which give a perfect explanation on solar neutrino problem. Since the neutrinos are changing flavors when traveling, the fraction of electron has changed to other flavors, and the fraction can be calculated from the mixing angles of oscillations. The mixing angles now have all been experimentally determined, with the last one $\theta_{13}$ found in 2012 in Daya Bay experiment~\cite{ping3}. However, the mixing angles and oscillation experiment only gives the relative mass between different flavors of neutrinos, this can not give the absolute mass of neutrinos. Hence the neutrino mass is crucial problem to understand nowadays. Neutrinos may be Majorana particles which is that they are their own antiparticles. If this is true, then not only the absolute mass of neutrinos can be observed, but also it explains why after the universe evolves, the matter dominates over anti-matter~\cite{ping4}.

%The neutrinoless double beta decay experiment are very similar to dark matter search (WIMP search). The EXO experiment[6] is the Enriched Xenon Observatory in particle physics looking into detection of ``neutrino-less double beta decay''. It uses large amounts of xenon isotopically enriched in the isotope 136. A total 200-kg detector using liquid Xe is currently being installed at the Waste Isolation Pilot Plant (WIPP) near Carlsbad, New Mexico. “Many research and development efforts are underway for a ton-scale experiment, with the goal of probing new physics and the mass of the neutrino.” (EXO experiment). The CoGeNT[7] Dark Matter Experiment is a direct search of dark matter particles in a low-background germanium detector located at Soudan Underground Laboratory in Soudan, MN. The detector uses a single, 440-gram, high-purity germanium crystal cooled to liquid nitrogen temperatures in its measurements. The CoGeNT detector has very low energy threshold (~0.5 keV) so that it takes its advantage to search for nuclear recoil events from dark matter particles with relatively low mass. The detector is also built with very low background so that could recognize the signals very efficiently. Another advantage of the CoGent detector is that it senses only ionization charge from nuclear recoils and places limits on the mass and interaction cross-section of dark matter particles by excluding any candidate mass and cross-section pair that would result in a signal above the background of the detector.This experiment began collecting data in December 2009. 

\subsection{Theory Justin}
xxxxxxxx xxxxxx xxxxxxxx xxxxxx xxxxxxxx xxxxxx xxxxxxxx xxxxxx xxxxxxxx xxxxxx xxxxxxxx xxxxxx xxxxxxxx xxxxxx
\section{Previous Experimental Efforts}
There exists considerable experimental evidence for the existence of a fourth neutrino at energy scales in the eV range. The sources for evidence include the short baseline neutrino oscillation experiments LSND (Liquid Scintillator Neutrino Detector) at Los Alamos~\cite{LSND}, and MiniBooNE (Mini Booster Neutrino Experiment) at Fermilab~\cite{mini1,mini2}. There is also the reactor neutrino anomaly~\cite{reactor_anom1}, and also in the radioactive source calibrations for the solar neutrino experiments GALLEX~\cite{gallex1,gallex2} and SAGE~\cite{sage1,sage2}.
\subsection{LSND}
The LSND experiment was developed to search for $\overline{\nu}_\mu \rightarrow \overline{\nu}_e$ oscillations. The source of $\overline{\nu}_\mu$ was the decay at rest (DAR) of $\mu^+$. LSND used a proton beam on a target to produce pions. The decay chain was as follows:
\begin{align}
  \begin{split}
    \pi^+ &\rightarrow \mu^+  \nu_\mu, \\
    \mu^+ &\rightarrow e^+  \nu_e  \overline{\nu}_\mu.
  \end{split}
\end{align}
The $\overline{\nu}_e$ events were detected through the interaction:
\begin{align}
  \overline{\nu}_e  p \rightarrow e^+ n,
\end{align}
inside of a large Cherenkov detector filled with liquid scintillator, utilizing 1280 phototubes. Figure~\ref{fig:lsnd_miniboone}L shows the excess of $\overline{\nu}_e$ events observed by LSND. The observation was $87.9\pm 22.4\pm6$ events over expected background. Their final result claimed the existence of $\overline{\nu}_\mu\rightarrow\overline{\nu}_e$ oscillations at $\Delta m^2 \sim 1\text{ eV}$ with 3.8$\sigma$. This is in heavy conflict with the existing three light flavor neutrino model supported by atmospheric and solar neutrino oscillation experiments, as well as measurements at LEP predicting $N=2.984\pm0.0082$ for weakly interacting neutrino flavors. Therefore, at least one ``sterile'' neutrino is required.
\begin{figure*}
  \centering
  \includegraphics[width=1\textwidth]{../figures/lsnd_miniboone.pdf}
  \caption{(L) From the LSND Experiment~\cite{LSND}. The expected number of events is represented by the red and green histograms, the blue histogram agrees with the data and models $\overline{\nu}_{\mu}\rightarrow \overline{\nu}_e$ oscillation. (C) The MiniBooNE electron event excess in neutrino and anti-neutrino sector~\cite{mini2}. (R) The MiniBooNE allowed regions in neutrino and anti neutrino mode, for a 2 neutrino model. Black stars represent the MiniBooNE best fit points for $\Delta m^2$ and $\sin^22\theta$~\cite{mini2}.}
  \label{fig:lsnd_miniboone}
\end{figure*}
\subsection{MiniBooNE}
The MiniBooNE experiment was developed to answer the LSND anomaly. The MiniBooNE detector also used Cherenkov radiation with photomultiplier tubes as their tool for detecting neutrino interactions. The Booster neutrino beam at Fermilab was used to create a $\nu_\mu$ or $\overline{\nu}_\mu$ beam. Protons were accelerated to 8 GeV by the Booster accelerator and were then steered to a beryillium target where a magnetic focusing horn steered pions from the proton on target collisions down the beam line towards the detector; the pions would decay into muons and muon neutrinos between the horn and the detector. Changing the magnetic focusing horn mode allowed for a beam of mostly $\nu_\mu$ or mostly $\overline{\nu}_\mu$ (neutrino or anti neutrino modes). In neutrino mode $\pi^+$ were steered towards the detector, while $\pi^-$ were steered away; and the reverse for anti neutrino mode. This is to utilize the decays:
\begin{align}
  \begin{split}
    \pi^+ &\rightarrow \mu^+\nu_\mu \\
    \pi^- &\rightarrow \mu^-\overline{\nu}_\mu  
  \end{split}
\end{align}
The full dataset from MiniBooNE was consistent with the LSND $\overline{\nu}_e$ excess, and therefore the sterile neutrino hypothesis survived. Figures~\ref{fig:lsnd_miniboone}C and \ref{fig:lsnd_miniboone}R show the electron excess and allowed oscillation regions from the MiniBooNE data, respectively.

The systematics of the MiniBooNE detector made it difficult to differentiate between $\pi^0 \rightarrow \gamma\gamma$ and an electron in the detector; therefore, there is a large uncertainty in the total number of $\overline{\nu}_e$ events. It is also hypothesized that the neutrino flux from the Booster neutrino beam must be studied more rigorously.
\subsection{Reactor Anomaly}
There has been an effort to re-analyse short baseline reactor neutrino oscillation data using newer flux calculations. These calculations show the possibility of the disappearance of reactor $\overline{\nu}_e$.
\begin{figure}[H]
  \centering
  \includegraphics[width=1\columnwidth]{../figures/reactor_anom.png}
  \caption{Plot for the short baseline reactor antineutrino anomaly. The red curve shows a possible three active neutrino mixing solution, using $\sin^2\theta_{13} = 0.06$. The blue curve shows a possible solution including a new neutrino mass state with $\Delta m^2 \gg 1\text{ eV}^2$.}
\end{figure}

\section{Current and Future Experimental Efforts}
Due to the number of anomalous results to date sterile neutrino searches have become popular research topics at the majority of contemporary neutrino experiments, both those currently running and those in the design and construction phases.

\subsection{Current Experiments}

\subsection{Current Experimental Results}


\subsubsection{MINOS/MINOS+}
The MINOS (Main Injector Neutrino Oscillation Search) experiment ran from 2005 to 2012 and its wide-band fresh, MINOS+, has taking data since September 2013. The MINOS experiment is a long baseline style detector with a neutrino beam source and near detector at Fermilab and a far detector located in northern Minnesota. Though its primary physics goals lie in the area of neutrino oscillation measurements, MINOS and MINOS+ are now also running sterile neutrino analysis with their data. MINOS detects sterile neutrinos via their disappearance in neutral-current (NC) interactions, which are independent of the three known neutrino flavors and their oscillations, and in charged-current (CC) interactions differing from those predicted from the mixing of the three neutrinos of the standard model. The latest sterile neutrino results from the collaboration \cite{MINOS} now exclude a large majority of the LSND allowed phase space, Figure~\ref{fig:MINOS}. These exclusion limits are predicted to increase by up to a factor of 2 as MINOS+ continues to run.

\begin{figure}[H]
  \centering
  \includegraphics[width=1\columnwidth]{../figures/minos2.pdf}
  \caption{MINOS sterile neutrino exclusion limits~\cite{MINOS}. MINOS is now excluding (to the right of the red line) much of the LSND allowed phase space (shaded).}
  \label{fig:MINOS}
\end{figure}

\subsubsection{Super-Kamiokande}
The Super-Kamiokande experiment is a 22.5 kton water Cherenkov detector that looks at atmospheric neutrino oscillations. Their latest sterile neutrino analysis~\cite{SuperK} used a sterile vacuum fit to put limits on the mixing between $\nu_{\mu}$ and a sterile neutrino $\nu_4$, parameterized by $\left|U_{\mu 4}\right|^2 \approx \sin^2 \theta_{24}$, Figure~\ref{fig:SuperK}. Super-K can not measure $\Delta m^{2}_{41}$ in this analysis however, so when comparing to MiniBooNE the Super-K result is just a straight line. The Super-K result is is consistent with the MiniBooNE exclusion, providing verification on at least part of the MiniBooNE excluded region.

\begin{figure}[H]
 \centering
 \includegraphics[width=1\columnwidth]{../figures/sk1.pdf}
 \caption{Super-Kamiokande sterile neutrino exclusion limits at 90\% (solid) and 99\% (dashed) confidence~\cite{SuperK}. The Super-K sterile vacuum fit can not measure $\Delta m^{2}_{41}$ so it is plotted as a straight line when compared to the joint analysis of MiniBooNE and SciBooNE (dotted).}
 \label{fig:SuperK}
\end{figure}

\subsubsection{Daya Bay}
The Daya Bay experiment is a reactor antineutrino experiment with six liquid scintillator detectors located at different baselines from six 2.9 GW$_\text{th}$ nuclear reactors. It is able to make sterile neutrino measurements at smaller mass squared differences, $10^{−3} \text{ eV}^2  \lesssim \left|\Delta m^{2}_{41}\right| \lesssim 0.1 \text{ eV}^2$, than have been probed by other experiments to date~\cite{DayaBay}. While they have many different combinations of their detectors data sets, the Daya Bay exclusion limits mark out more of the small mass difference phase space than the combined LSND and KARMEN limits.

\begin{figure}[H]
 \centering
 \includegraphics[width=1\columnwidth]{../figures/daya1.pdf}
 \caption{Daya Bay sterile neutrino exclusion limits~\cite{DayaBay}.}
 \label{fig:DayaBay}
\end{figure}


\subsection{Future Experiments}
% Matt
While there are many future neutrino experiments in various stages of development, MicroBooNE and NOvA are the closest to producing new sterile neutrino results. As its name suggests, MicroBooNE is the larger follow up experiment to MiniBooNE. The MicroBooNE detector is an advanced time projection chamber with a 60 ton fiducial volume of liquid argon, and unlike MiniBooNE it will be able to distinguish between photons and electrons. It is widely expected that these significant improvements will allow MicroBooNE to resolve the MiniBooNE excess anomaly. The NOvA experiment is long baseline experiment that uses the same NuMI neutrino beam from Fermilab as MINOS and also has its far detector located in northern Minnesota. The 330 ton near and 14 kton far detectors are liquid scintillator designs located near the surface. NOvA's primary purpose is to make oscillation measurements, including potentially determining the ordering of the mass hierarchy, but will produce more sterile neutrino measurements as well.

\section{Conclusions}
xxxxxxxx xxxxxx xxxxxxxx xxxxxx xxxxxxxx xxxxxx xxxxxxxx xxxxxx xxxxxxxx xxxxxx xxxxxxxx xxxxxx xxxxxxxx xxxxxx

\begin{thebibliography}{9}
\bibitem{LSND}
  A.~Aguilar-Arevalo \emph{et al.} Evidence for neutrino oscillations from the observation of anti-neutrino(electron) appearance in a anti-neutrino(muon) beam. \emph{Phys. Rev. D.}, {\bf 64} 112007, 2001.
\bibitem{mini1}
  A.~Aguilar-Arevalo \emph{et al.} Event Excess in the MiniBooNE Search for $\nu_\mu \rightarrow \nu_e$ Oscillations. \emph{Phys. Rev. Lett.}, {\bf 105} 181801, 2010.
\bibitem{mini2}
  A.~Aguilar-Arevalo \emph{et al.} Improved Search $\nu_\mu \rightarrow \nu_e$ Oscillations in the MiniBooNE Experiment. \emph{Phys. Rev. Lett.}, {\bf 110} 161801, 2013.
\bibitem{reactor_anom1}
  G. Mention, M. Fechner, Th. Lasserre, Th.A. Mueller, D. Lhuillier, \emph{et al.} The Reactor Antineutrino Anomaly. \emph{Phys. Rev. D.}, {\bf 83} 073006, 2011.
\bibitem{reactor_anom2}
  Th. A. Mueller, D. Lhuillier, M. Fallot, A. Letourneau, S. Cormon, \emph{et al.} Improved Predictions of Reactor Antineutrino Spectra. \emph{Phys. Rev. C.}, {\bf 83} 054615, 2011.
\bibitem{gallex1}
  P.~Anselmann~\emph{et al.} First results from the Cr-51 neutrino source experiment with the GALLEX detector. \emph{Phys. Lett. B.}, {\bf 342} 440-450, 1995
\bibitem{gallex2}
  W.~Hampel~\emph{et al.} Final results of the Cr-51 neutrino source experiments in GALLEX. \emph{Phys. Lett. B.}, {\bf 420} 114-126, 1998.
\bibitem{sage1}
  Dzh.~N.~Abdurashitov, V.N.~Gavrin, S.V. Girin, V.V. Gorbachev, Tatiana V. Ibragimova, \emph{et al.} The Russian-American gallium experiment (SAGE) Cr neutrino source measurement. \emph{Phys. Rev. Lett.}, {\bf 77} 4708-4711, 1996.
\bibitem{sage2}
  J.~N.~Abdurashitov~\emph{et al.} Measurement of the response of the Russian-American gallium experiment to neutrinos from a Cr-51 source. \emph{Phys. Rev. C.}, {\bf 59} 2246-2263, 1999.
\bibitem{MINOS}
  Alexandre B. Sousa (for the MINOS and MINOS+ Collaborations) First MINOS+ Data and New Results from MINOS. arXiv:1502.07715v1 [hep-ex] \url{http://arxiv.org/abs/1502.07715}
\bibitem{SuperK}
  K. Abe~\emph{et al.} (Super-Kamiokande Collaboration) Limits on sterile neutrino mixing using atmospheric neutrinos in Super-Kamiokande. \emph{Phys. Rev. D}, {\bf 91}, 052019, 2015, \url{http://dx.doi.org/10.1103/PhysRevD.91.052019}
\bibitem{DayaBay}
 F. P. An~\emph{et al.} (Daya Bay Collaboration) Search for a Light Sterile Neutrino at Daya Bay. \emph{Phys. Rev. Lett.} {\bf 113}, 141802, 2014, \url{http://dx.doi.org/10.1103/PhysRevLett.113.141802}

\end{thebibliography}

\end{document} %%% end of doc %%%
